\chapter*{Summary}

% Introduction to Dimensionality Reduction

% Summary of the introduction to Dimensionality Reduction Techniques from next chapter

% SLMVP
SLMVP as a supervised local technique, has a great advantage over the other techniques.

% Introduction to interpreting coefficients
Since the new dimensions found by the techniques are the directions of maximum variability of the data, by computing the correlation between the original data and each of the components, we can draw meaningful qualitative information about the components. The variables that are most strongly correlated with each component will be the ones that the technique deems as more significant.

Furthermore, we can compare two components obtained with two different dimensionality reduction techniques on basis of their interpretability. We can do this by calculating the spearman correlation coefficient of the absolute correlation between the components.

% Classification
In this work, a brief evaluation of the techniques on basis of their classification results is also offered. A handful of machine learning classifiers are trained on the data to see how SLMVP and other techniques compare in a classification task.

% SLMVP
The results show that SLMVP can more meaningfully separate the data into clusters based on the values of the dependent variable. This makes for more visually impactful 2-dimension plots, with cluster that are more clearly separated than when using other techniques. Moreover, SLMVP is more robust than other techniques that perform on par.

%%--------------
\newpage
%%--------------

\chapter*{Abstract}

<<Abstract of the Master Project. Maximum length: 2 pages.>>


%%%%%%%%%%%%%%%%%%%%%%%%%%%%%%%%%%%%%%%%%%%%%%%%%%%%%%%%%%%
%% Final del resumen. 
%%%%%%%%%%%%%%%%%%%%%%%%%%%%%%%%%%%%%%%%%%%%%%%%%%%%%%%%%%%